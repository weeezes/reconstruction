\documentclass[12pt,a4paper]{article}
\usepackage[utf8x]{inputenc}
\usepackage{ucs}
\usepackage[finnish]{babel}
\usepackage[T1]{fontenc}
\usepackage{amsmath}
\usepackage{amsfonts}
\usepackage{amssymb}
\usepackage{graphicx}
\usepackage{hyperref}
\usepackage[figurename=Kuva, tablename=Taulukko]{caption}

\author{Vesa Hagström}


\begin{document}
    \renewcommand{\bibname}{Viitteet}

	\begin{titlepage}
    \label{Title}
	\begin{flushleft}
		\hfill Vesa Hagström 											\\
		\hfill \texttt{vesa.hagstrom@helsinki.fi} 						\\
		\hfill Op.Nro: 013865575											\\
		\hfill Tiralabra harjoitustyö									\\
		\hfill 16.6.2013													\\
	\end{flushleft}

	\vfill

	\begin{center}
		\huge{Simrec testausdokumentti}
	\end{center}

	\vfill

	\end{titlepage}

    \section{Yksikkötestit}
    \label{yksikko}

    Ohjeet yksikkötestien ajamiseen löytyy dokumentista \textit{Simrec käyttöohje}.
    
    Yksikkötesteillä testataan olennaiset asiat ohjelman ja sen luokkien toiminnallisuuksista. Kattava ja ajantasainen kooste yksikkötesteistä löytyy vilkaisemalla \textit{tests}-kansiosta löytyvää lähdekoodia.
    
    \section{Käsintestaus}
    Kansiosta \textit{misc} löytyy testidataa, joka on nimetty siten että tiedostonimestä käy ilmi kuvan dimensiot. Kuvien nimet ovat muodossa \texttt{tunniste\_leveys\_korkeus\_bittisyys.raw}. Testidata on nimetty siten, että tunnisteesta käy ilmi, mitä tuloksen tulisi muistuttaa. 
    
    Käsintestauksen askeleet:
    \begin{enumerate}
    \item Aja simrec antamalla sille jokin \textit{misc}-kansion testikuvista. Ohjeet dokumentissa \textit{Simrec käyttöohje}.
    \item Avaa esimerkiksi ImageJ:llä simrec:n tuloskuva \textit{transformed.raw} käyttämällä raakadatan import-toimintoa. Simrecin tulokset ovat 32-bittisiä liukulukuja, tuloksen dimensiot käyvät ilmi simrec:n ajon päättyessä tulostamasta tekstistä.
    \end{enumerate}
    
    Kansiosta \textit{doc/valmiita\_tuloksia} löytyy png-muotoisia kuvia, mistä näkee sinogrammin ja siitä simrec:llä muodostetun tuloksen.
\end{document}