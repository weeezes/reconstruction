\documentclass[12pt,a4paper]{report}
\usepackage[utf8x]{inputenc}
\usepackage{ucs}
\usepackage[finnish]{babel}
\usepackage[T1]{fontenc}
\usepackage{amsmath}
\usepackage{amsfonts}
\usepackage{amssymb}
\usepackage{graphicx}
\usepackage{hyperref}
\usepackage[figurename=Kuva, tablename=Taulukko]{caption}

\author{Vesa Hagström}

\begin{document}
    \renewcommand{\bibname}{Viitteet}

	\begin{titlepage}
    \label{Title}
	\begin{flushleft}
		\hfill Vesa Hagström 											\\
		\hfill \texttt{vesa.hagstrom@helsinki.fi} 						\\
		\hfill Op.Nro: 013865575											\\
		\hfill Tiralabra harjoitustyö									\\
		\hfill 8.6.2013													\\
	\end{flushleft}

	\vfill

	\begin{center}
		\huge{4. Viikkoraportti}
	\end{center}

	\vfill

	\end{titlepage}

    \section*{Selvitys}
    Neljännellä viikolla en tehnyt mitään. 7. ja 8. päivä yritin saada tiedostoon kirjoittamisen toimimaan, ja 8. päivä jopa onnistuin saamaan jotakin tallennettua. 
    Onnistun joillakin tyypeillä lukemaan kuvan tiedostosta, Fourier-muuntamaan kuvan, kääntesmuuntamaan sen takaisin ja kirjoittamaan tuloksen kuvatiedostoksi hyvin tuloksin. Ilmeisesti näiden tyyppien arvoalue ei kuitenkaan jostakin syystä riitä itse Radon-muunnoksen käänteismuunnoksen tallentamiseen, vaikka en oleta tuloksessa olevan massiivisia arvoja. Selvästi sekä Radon-käänteismuunnokseni että arvojen tyyppimuunnokset ovat rikki. 
    Kaikkea koodia ei ole testattu eikä kommentoitu, jätän tämän tehtäväksi - piru minut periköön - seuraavalle viikolle, ja saada tämän viikon edes jotenkuten pakettiin.  
    
    \section*{Mitä seuraavaksi?}
    Seuraavaksi olisi syytä saada selvitettyä kuinka data tallennetaan tiedostoon niin, että mikään ei hajoa matkalla. Intresseihini kuuluu myös saada itse Radon-käänteismuunnos valmiiksi. Minulla on kumpaankin jonkinlainen runko aluilla, ongelmaksi koituu kuitenkin se, että kun toinen on hajalla niin on hankalaa todentaa toisen toimivan. En keksi tapaa varmistaa Radon-käänteismuunnoksen toimivuutta muuta kuin silmämääräisesti tarkistamalla lopputuloksen.    
     
\end{document}