\documentclass[12pt,a4paper]{report}
\usepackage[utf8x]{inputenc}
\usepackage{ucs}
\usepackage[finnish]{babel}
\usepackage[T1]{fontenc}
\usepackage{amsmath}
\usepackage{amsfonts}
\usepackage{amssymb}
\usepackage{graphicx}
\usepackage{hyperref}
\usepackage[figurename=Kuva, tablename=Taulukko]{caption}

\author{Vesa Hagström}

\begin{document}
    \renewcommand{\bibname}{Viitteet}

	\begin{titlepage}
    \label{Title}
	\begin{flushleft}
		\hfill Vesa Hagström 											\\
		\hfill \texttt{vesa.hagstrom@helsinki.fi} 						\\
		\hfill Op.Nro: 013865575											\\
		\hfill Tiralabra harjoitustyö									\\
		\hfill 30.5.2013													\\
	\end{flushleft}

	\vfill

	\begin{center}
		\huge{3. Viikkoraportti}
	\end{center}

	\vfill

	\end{titlepage}

    \section*{Selvitys}
    Käytin kolmannen viikon pääasiassa koodin refaktoroimiseen. Aloitin myös luokkien ja metodien kommentoimisen. Käytän koodin API:n dokumentin luomiseen Doxygen-työkalua, mikä tuottaa koodin kommenteista hieman samankaltaisen tuloksen kuin minkä JavaDoc tekee. 
        
    Sain tällä viikolla myös implementoitua algoritmin FFT:n laskemiselle kaksiulotteisesta datasta. 2D FFT voidaan laskea suorittamalla datalle 1D FFT ensin riveittäin, sitten sarakkeittain. Tämä voidaan tehdä separoituvalle funktiolle, koska näille voidaan osoittaa seuraava:
    
    
    \begin{equation*}
        g(x,y) = g_{x}(x)g_{y}(y)
    \end{equation*}

    \begin{equation*}
        \mathcal{F}[g(x,y)] = \mathcal{F}_{x}[g(x)]\mathcal{F}_{y}[g(y)]
    \end{equation*}
    
    Olin ensin toteuttanut 2D FFT:ni väärin, koska en ollut osannut luoda testeilleni testidataa oikein. Ihmettelin pitkään miksi vääräksi korjaamani toteutus toimi, eikä teorian mukainen toiminut, mutta onneksi virhe löytyi. 
    
    Toteutin myös käänteismuunnoksen FFT:lleni. Tässä hauskaksi ominaisuudeksi osoittui Fourier -käänteismuunnoksen ominaisuus verrattuna Fourier -muunnokseen: ainoastaan kaksi -1 kerrointa FFT algoritmissa tarvitsee muuttaa positiivisiksi iFFT:n laskemiseksi.
    
    
    \section*{Mitä seuraavaksi?}
    Nyt kun koodi on melko siistiä ja kommentointi liki ~100\% kattavaa, ja minulla on valmiina kaikki FFT:hen liittyvä, voin siirtyä toteuttamaan Filtered Back Projection -algoritmia. Tavoitteenani oli päästä toteuttamaan tätä jo menneellä viikolla, mutta ongelmat 2D FFT:ni kanssa hieman hidastivat. Halusin myös saada häiritsevät refaktorointia kaipaavat kohdat koodista siistittyä. Nyt alkaa vihdoin näyttämään siltä, että projektista voi jopa syntyä jotakin valmista!
     
\end{document}